\chapter{Потенциальные функции в частичных областях}

Система уравнений Максвелла в двумерном случае
\[  \frac{\partial H_y}{\partial z} = i\omega\varepsilon_0\varepsilon E_x , \quad 
\frac{\partial E_y}{\partial z} = -i\omega\varepsilon_0\varepsilon E_x
 ,\]
 \[  \frac{\partial H_x}{\partial z} -
  \frac{\partial H_z}{\partial x} =
 i\omega\varepsilon_0\varepsilon E_x , \quad 
\frac{\partial E_x}{\partial z} -
\frac{\partial E_z}{\partial x}=
i\omega\varepsilon_0\varepsilon E_x
, \]
\[  \frac{\partial H_y}{\partial x} = -i\omega\varepsilon_0\varepsilon E_x , \quad 
\frac{\partial E_y}{\partial x} = i\omega\varepsilon_0\varepsilon E_x
 ,\]
распадается на две независимые подсистемы.

Для ТE-волн
$$
	E_y=u, H_x= \frac{\partial u}{\partial z} \frac{1}{i\omega\mu_0\mu}
$$,
$$
	H_z=- \frac{\partial u}{\partial x} \frac{1}{i\omega\mu_0\mu}
$$

Потенциальная функция $u(x,z)$ удовлетворяет уравнению Гельмгольца 
$$ \frac{\partial^2 u}{\partial x^2} +
\frac{\partial^2 u}{\partial z^2} +
k^2u =0, k^2=\omega^2\mu_0\mu\varepsilon_0\varepsilon
$$.

На стенках волновода должны быть равны нулю касательные составляющие вектора $E$, то есть компоненты $E_y$ и $E_z$.

Исследуем элементарные ТE-волны плоского волновода. Для этого найдем методом разделения переменных частные решения уравнения Гельмгольца в полосе $0<x<a$, удовлетворяющие условиям $u=0$ при $x=0$ и при $x=a$. Если $u(x,z)= X(x)Z(z), $ то

$$
	\frac{X''(x)}{X(x)}+ \frac{Z''(z)}{Z(z)}+k^2=0
$$

Введем постоянную разделения, равную первой дроби в левой части последнего равенства. При неотрицательных значениях этой постоянной не существует ненулевых решений уравнения $ X''(x)- \alpha^2 X(x)=0$, удовлетворяющих краевым условиям $X'(0)=0, X'(a)=0 $. Если же постоянная разделения отрицательная, то уравнение $ X''(x)- \alpha^2 X(x)=0$ имеет ненулевые решения $X_m(x)= c\cos\alpha_mx$, удовлетворяющие краевым условиям, при $\alpha=\alpha_m=\pi m a, m=0,1,...$. Пусть 
$\gamma_m=\sqrt{k^2-\left(\frac{\pi m}{a}\right)^2}$,

тогда потенциальным функциям 
$ u_m(x,z)=c\sin\frac{\pi m x}{a}\exp^{\pm i\gamma_m z}, m=0,1,...\quad , $
соответствуют ТМ-волны плоского волновода. Знак в показателе экспоненты выбирается в зависимости от направления волны.

Потенциальную функцию приходящей слева волны зададим в виде
$$
	u^0(x, z) = a_l^0 e^{i\gamma_{al}z}\sin{\frac{\pi l}{a}x}
$$
и будем искать потенциальные функции рассеянного поля в частичных областях в виде
$$
	\begin{array}{ll}
	u^1(x, z)=\sum\limits_{n=1}^{\infty}b_n^1e^{-i\gamma_{an}z}\sin{\frac{\pi n}{a}x},&
	u^2(x, z)=\sum\limits_{n=1}^{\infty}a_n^2e^{i\gamma_{an}z}\sin{\frac{\pi n}{a}x},\\
	u^3(x, z)=\sum\limits_{n=1}^{\infty}b_n^3e^{-i\gamma_{bn}z}\sin{\frac{\pi n}{b-a}(x-a)},&
	u^4(x, z)=\sum\limits_{n=1}^{\infty}a_n^4e^{i\gamma_{bn}z}\sin{\frac{\pi n}{b-a}(x-a)},\\
	u^5(x, z)=\sum\limits_{n=1}^{\infty}\left[a_n^5e^{i\gamma_{n}z}+b_n^5e^{-i\gamma_{n}z}\right]\sin{\frac{\pi n}{b}x}.&\\
	\end{array}
$$