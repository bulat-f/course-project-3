\chapter{Постановка задачи}

Имеется плоский волновод с проводящими стенками. Внутри волновода размещена продольная перегородка с отверстием. Т. е. две бесконечные полуплоскости, паралельные стенкам, разделяют волновод естсвенным образом на пять частей.

(1) = $\{0<x<a, z<0\}$
(2) = $\{0<x<a, z>c\}$
(3) = $\{a<x<b, z<0\}$
(4) = $\{a<x<b, z>c\}$
(5) = $\{0<x<b, 0<z<c\}$

Выберем направление осей декартовой системы координат так, как показано на рисунке.

\begin{tikzpicture}[scale=1]
	% coord system
	\draw[->] (-3,0)--(6,0) node[anchor=north] {$z$};
	\draw[->] (0,3)--(0,4) node[anchor=east] {$x$};
	
	\node at (0.15,-0.15) {$0$};
	\node at (0.15,1.5) {$a$};
	\node at (0.15,3.15) {$b$};
	\node at (2.15,-0.15) {$c$};
	
	\node at (-0.25,1.25) {$1$};
	\node at (-0.25,2.75) {$3$};
	\node at (2.25,1.25) {$2$};
	\node at (2.25,2.75) {$4$};
	\node at (1,1.5) {$5$};
	
	\draw[dashed] (0,0)--(0,3);
	\draw[dashed] (2,0)--(2,3);

	\draw (-3,1.5) -- (0,1.5);
	\draw (-3,3) -- (4.5,3);
	\draw (2,1.5) -- (4.5,1.5);
	
	\draw[->] (-2,0.5)--(-0.5,0.5) node[anchor=north] {$u^0$};
	\draw[->] (-1,1)--(-2.5,1) node[anchor=north] {$u^1$};
	\draw[->] (-1,2.25)--(-2.5,2.25) node[anchor=north] {$u^3$};
	
	\draw[->] (3,0.75)--(4.5,0.75) node[anchor=north] {$u^2$};
	\draw[->] (3,2.25)--(4.5,2.25) node[anchor=north] {$u^4$};
\end{tikzpicture}

Будем рассматривать гармоническое TE-поляризованное двумерное электромагнитное поле, компоненты которого не зависят от координаты $y$. Зависимость поля от времени выберем в виде $e^{-i\omega t}$.

Пусть из области (1) набегает собственная волна с номером $l$.

Надо найти электромагнитное поле, возникающее при ее дифракции.