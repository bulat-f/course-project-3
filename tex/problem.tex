\section{Постановка задачи}
Имеется плоский волновод с проводящими стенками. Внутри волновода размещена продольная перегородка с отверстием. С нижней левой стороны (отмечена номером 1) набегает плоская волна параллельной поляризации.

\begin{tikzpicture}[scale=1]
	% coord system
	\draw[->] (-3,0)--(6,0) node[anchor=north] {$z$};
	\draw[->] (0,-1)--(0,4) node[anchor=east] {$x$};
	
	\draw (-3,1.5) -- (0,1.5);
	\draw (-3,3) -- (4.5,3);
	\draw (2,1.5) -- (4.5,1.5);
	
	\draw[->] (-2.5,0.75)--(-1,0.75) node[anchor=north] {$u^0$};
	\draw[->] (-1,2.25)--(-2.5,2.25) node[anchor=north] {$u^3$};
	
	\draw[->] (3,0.75)--(4.5,0.75) node[anchor=north] {$u^2$};
	\draw[->] (3,2.25)--(4.5,2.25) node[anchor=north] {$u^4$};
\end{tikzpicture}

Дано
$$
	u^0(x, z) = a_l^0 e^{i\gamma_{al}z}\sin{\frac{\pi l}{a}x}
$$